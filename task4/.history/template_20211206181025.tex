\documentclass[fontsize=11pt]{scrartcl}
%-------------------------------------------------------------------------------------------------
% LATEX TEMPLATE FOR A BACHELOR'S THESIS AT GHENT UNIVERSITY GLOBAL CAMPUS
% CREATED BY MANVEL GASPARYAN
% APRIL, 2021
%-------------------------------------------------------------------------------------------------
\usepackage{hyperref, tikz, float, subfigure, multicol, amsmath, amsthm, alphalph, amsfonts, amssymb, geometry, enumitem, parskip, xcolor, sectsty}
\usepackage[normalem]{ulem}
\usepackage[font=scriptsize,labelfont=bf]{caption}
\usepackage[explicit]{titlesec}
\usepackage[scaled]{helvet}
\renewcommand\familydefault{\sfdefault} 
\usepackage[T1]{fontenc}

\usepackage{setspace}

\usepackage{{listings}}

% \renewcommand{\contentsname}{}

%-------------------------------------------------------------------------------------------------
 \geometry{a4paper, total={170mm,257mm}, left=20mm, right=20mm, top=25mm, bottom=30mm}
%-------------------------------------------------------------------------------------------------
\newtheorem{proposition}{Proposition}[section]
\newtheorem{lemma}{Lemma}[section]
\newtheorem{remark}{Remark}[section]
\newtheorem{corollary}{Corollary}[section]
\newtheorem{definition}{Definition}[section]
%-------------------------------------------------------------------------------------------------
\pagenumbering{roman}
\usepackage{fancyhdr}
\pagestyle{fancy}
\fancyhf{}
\renewcommand{\headrulewidth}{0pt}
\rhead{}
\lhead{}
\rfoot{\thepage}
\lfoot{}

%-------------------------------------------------------------------------------------------------
\definecolor{ghent_blue}{rgb}{0.1176, 0.392, 0.7843}
\definecolor{ghent_dark}{rgb}{0.0, 0.2, 0.4}
%-------------------------------------------------------------------------------------------------
\title{{\color{ghent_blue} TITLE: LATEX TEMPLATE FOR A BACHELOR'S THESIS AT GHENT UNIVERSITY GLOBAL CAMPUS}}
\subtitle{{\color{ghent_dark} [DOCUMENT SUBTITLE]}}
\date{}         
%-------------------------------------------------------------------------------------------------
\sectionfont{\fontsize{16}{15}\selectfont}
\sectionfont{\color{ghent_blue}}
%=================================================================================================
\begin{document}
%=================================================================================================
%PAGE i: TITLE PAGE 1
%-------------------------------------------------------------------------------------------------
\thispagestyle{empty}
\hfill\includegraphics[scale = 1]{img/badge/2021.png}\\
%-------------------------------------------------------------------------------------------------
\vspace{8cm}

\noindent{\fontsize{30}{50}\selectfont{\color{ghent_blue}\noindent \hspace{12mm}\textbf{
Microphotonics %TITLE
}}}\\
\fontsize{20}{50}\selectfont{\color{ghent_dark}\noindent \hspace{13mm}\textbf{
CAD-LAB: Periodic Structure%SUBTITLE
}}\vspace{10mm}

\hspace{10mm}{\fontsize{10}{10}\selectfont{\textbf{
Lukuan Zhang, Rui Zhu, Xiyuan Guo%AUTHOR
}}}
\vspace*{\fill}
%-------------------------------------------------------------------------------------------------
\begin{flushleft}
\begin{figure}[b!]
\includegraphics[scale = 1.4]{img/badge/GUGC.pdf}
\end{figure}
\end{flushleft}

\doublespacing
\tableofcontents
\pagebreak
\pagenumbering{arabic}



%=================================================================================================
\pagebreak
%=================================================================================================
%TASK 1
\section{\uline{Surface Grating}}
%-------------------------------------------------------------------------------------------------
%-------------------------------------------------------------------------------------------------
% Here comes some text. This text makes use of 1.5 line spacing. 
%-------------------------------------------------------------------------------------------------
\subsection{}
%-------------------------------------------------------------------------------------------------
From Figure \ref{fig1.1} of the $k$-vector diagram of the surface grating, we get
\begin{equation}
    K=\frac{2 \pi}{\Lambda}=\frac{k_{0} n}{\sqrt{2}}
    \end{equation}
\begin{equation}
    \Lambda=\frac{2 \pi \sqrt{2}}{k_{0} n}=917 \mathrm{~nm}
    \end{equation}   
So the period of surface grating is $0.917\mathrm{\mu m}$.
\begin{figure}[H]
    \centering
     \includegraphics[width=0.4\textwidth]{img/fig1.1.png}
     \caption{The $k$-vector diagram of the surface grating}
     \label{fig1.1}
\end{figure}
%-------------------------------------------------------------------------------------------------
\subsection{}
%-------------------------------------------------------------------------------------------------
Set $D$ as $0.1\mathrm{\mu m}$ and period $\Lambda$ as $0.917\mathrm{\mu m}$ and 
run the simulation, then we can get Figure \ref{fig1.2}, 
which shows the plane wave mainly propagates in the $0^\circ$ direction, 
and partly propagates in the $45^\circ$ and $-45^\circ$ direction.
\begin{figure}[H]
    \centering
     \includegraphics[width=0.45\textwidth]{img/fig1.2.png}
     \caption{Distribution of E2 in farfield with 
     period $0.917\mathrm{\mu m}$ and $D$ $0.1\mathrm{\mu m}$}
     \label{fig1.2}
\end{figure}
%-------------------------------------------------------------------------------------------------
\subsection{}
%-------------------------------------------------------------------------------------------------
\textbf{1. Increase the diffraction to the $45^\circ$ orders}

Getting the geometic relationship from Figure \ref{fig1.3}, 
an analytical expression can be figured out 
from 
\begin{equation}
    k_{0}\left(\sqrt{2} D n-n_{0} D\right)=2 \pi m 
\end{equation} 
For $\lambda=1.55\mathrm{\mu m}$, $n=2.39$, $n_0=1$ and $m=1$, we get $D=0.6513\mathrm{\mu m}$. 
Adjusting the parameters and running the simulation again, 
we can see an apparent improvement in Figure \ref{fig1.4}
\begin{figure}[H]
    \centering
     \includegraphics[width=0.4\textwidth]{img/fig1.3.png}
     \caption{The surface gating diagram}
     \label{fig1.3}
\end{figure}
\begin{figure}[H]
    \centering
     \includegraphics[width=0.45\textwidth]{img/fig1.4.png}
     \caption{Distribution of E2 in farfield with 
     period $0.917\mathrm{\mu m}$ and $D$ $0.6513\mathrm{\mu m}$}
     \label{fig1.4}
\end{figure}

\textbf{2. Decrease the $0^\circ$ diffraction}

An analytical expression can be figured out from
\begin{equation}
    k_{0}\left((D+\Lambda) n-n_{0} D\right)=2 \pi m
\end{equation}
For $\lambda=1.55\mathrm{\mu m}$, $\Lambda=0.917\mathrm{\mu m}$, 
$n=2.39$, $n_0=1$ and $m=1$, we get $D=0.6536\mathrm{\mu m}$. 
Adjusting the parameters and running the simulation again, 
we can also see an apparent improvement in Figure \ref{fig1.6}.
\begin{figure}[H]
    \centering
     \includegraphics[width=0.45\textwidth]{img/fig1.6.png}
     \caption{Distribution of E2 in farfield with 
     period $0.917\mathrm{\mu m}$ and $D$ $0.6536\mathrm{\mu m}$}
     \label{fig1.6}
\end{figure}


%=================================================================================================
\pagebreak
%=================================================================================================
%TASK 2
%-------------------------------------------------------------------------------------------------
\section{\uline{Distributed Bragg Refector}}
According to the $k$-diagram in Figure \ref{k-dia-BR}, we can calculate the period required,
\begin{equation}
    \Lambda=\frac\lambda{2n_{neff}}=0.322917\mu m
\end{equation}
\begin{figure}[H]
    \centering
    \includegraphics[width=0.7\textwidth]{img/k-dia-BR.png}
    \caption{$\kappa$ vector diagram}
    \label{k-dia-BR}
\end{figure}
\begin{figure}[H]
    \centering
    \includegraphics[width=0.35\textwidth]{img/n2.4_mov.png}
    \hspace{10mm}
    \includegraphics[height=0.4\textwidth]{img/n2.4_T.png}
    \caption{Propagation process(Left). The transmission spectrum of the grating(Right) }
    \label{n2.4_porpa}
\end{figure}
Adjusting it in the \textit{WaveguideGrating.fsp} simulation file,
the result is shown in Figure \ref{n2.4_propa}. Obviously, it is not what we expect.
The smallest transmission is not at $\lambda=1.55\mu m$. We think it's because the air surounding and boundary condition or other
reasons that have influences on the effective refractive index. If the simulation is real,
the real $n_{eff}$ should meet \\
\begin{equation}
    \frac{\lambda '}{2n_{eff }'}=\frac{\lambda }{2n_{eff}}
\end{equation}
Then we get $n_{eff}=2.346$ and $\Lambda=0.3304$. In the simulation, we get the reflectivity 0.972.
According to formula $(6.87)$, we calculate $\kappa$
\begin{equation}
    \kappa=\frac 1{2N\Lambda}\ln{\frac{1+\sqrt R}{1-\sqrt R}}=0.250  {\mu \mathrm{m}}^{-1}
\end{equation}
\begin{figure}[H]
    \centering
    \includegraphics[width=0.4\textwidth]{img/n2.3._Tpng.png}
    \hspace{10mm}
    \includegraphics[height=0.4\textwidth]{img/R.png}
    \caption{The transmission spectrum of the grating when $n_{eff}=2.346$(Left). The reflectivity(Right) }
    \label{n2.4_porpa}
\end{figure}
After changing $N$, we could get the relation between $\kappa L$ and $R$, as shown in Figure \ref{Rk}.
Obviously, $L$ is proportional to $\tanh^{-1}{\sqrt R}$. Simulating it with Matlab, we get $\kappa=0.2292\mu m^{-1}$.
\begin{figure}[H]
    \centering
    \includegraphics[width=0.5\textwidth]{img/kappa.png}
    \caption{Relation between $\tanh^{-1}{\sqrt R}$ and $L$. }
    \label{Rk}
\end{figure}
\pagebreak
%=================================================================================================
%TASK 3
%-------------------------------------------------------------------------------------------------
\section{\uline{Grating Coupler}}
%-------------------------------------------------------------------------------------------------
% Here comes some text. This text makes use of 1.5 line spacing. 
%-------------------------------------------------------------------------------------------------
\subsection{}
%-------------------------------------------------------------------------------------------------
From Figure \ref{fig3.1} of the $k$-vector diagram of the waveguide grating, 
the period of waveguide grating is calculated as $0.645\mathrm{\mu m}$.
\begin{equation}
    K=\frac{2 \pi}{\Lambda}=k_{i n}=\frac{2 \pi}{\lambda} n_{e f f}
\end{equation}
\begin{equation}
    \Lambda=\frac{\lambda}{n_{\text {eff }}}=\frac{1550 \mathrm{~nm}}{2.4}=645.8 \mathrm{~nm}
\end{equation}  
\begin{figure}[H]
    \centering
     \includegraphics[width=0.7\textwidth]{img/fig3.1.jpg}
     \caption{The $k$-vector diagram of the waveguide grating}
     \label{fig3.1}
\end{figure}
%-------------------------------------------------------------------------------------------------
\subsection{}
%-------------------------------------------------------------------------------------------------
Figure \ref{fig3.2} shows transmission spectrum when the light propagates upwards. 
There is a dip around $1550\mathrm{nm}$, not as expected. 
In theory, there may be a peak instead of a dip at about $1550\mathrm{nm}$. 
The reason is that $n_{eff}$ is smaller than the theoretical value $2.4$ 
for the periodic structure. 
The effective refractive index of unetched part waveguide is $2.4$, 
but the etched part is smaller, 
so the effective refractive index of the whole waveguide is smaller than $2.4$.
\begin{figure}[H]
    \centering
     \includegraphics[width=0.45\textwidth]{img/fig3.2.png}
     \caption{Transmission spectrum of monitor\_up at $\Lambda=645.8\mathrm{nm}$}
     \label{fig3.2}
\end{figure}
%-------------------------------------------------------------------------------------------------
\subsection{}
%-------------------------------------------------------------------------------------------------
Figure \ref{fig3.3} shows the far-field radiation pattern of upward. 
The maximum of $E$ is only $1.5\times 10^{-7}$. 
By varying the wavelength, we found that $E$ reaches the maximum $7.3\times 10^{-7}$ 
at $\lambda=1468.7\mathrm{nm}$, as Figure \ref{fig3.4} shown. This wavelength also corresponds to 
the peak of transmission spectrum when the light propagates upwards, 
shown as Figure \ref{fig3.5}.
\begin{figure}[H]
    \centering
     \includegraphics[width=0.45\textwidth]{img/fig3.3.png}
     \caption{Far-field radiation pattern of upward at $\Lambda=645.8\mathrm{nm}$,
     $\lambda=645.8\mathrm{nm}$ and $n_{eff}=2.4$}
     \label{fig3.3}
\end{figure}
\begin{figure}[H]
    \centering
     \includegraphics[width=0.45\textwidth]{img/fig3.4.png}
     \caption{Far-field radiation pattern of upward at $\Lambda=645.8\mathrm{nm}$,
     $\lambda=1468.7\mathrm{nm}$ and $n_{eff}=2.4$}
     \label{fig3.4}
\end{figure}
\begin{figure}[H]
    \centering
     \includegraphics[width=0.45\textwidth]{img/fig3.5.png}
     \caption{The peak of upward transmission spectrum at $\Lambda=645.8\mathrm{nm}$,
     $\lambda=1468.7\mathrm{nm}$}
     \label{fig3.5}
\end{figure}
%=================================================================================================
\subsection{}
The method to correct the dip in the trasmission spectrum is 
to enlarge $\Lambda$, because the real $n_{eff}$ is smaller than the theoretical value $2.4$.
Figure \ref{fig3.6} shows that the upward transmission spectrum for $\Lambda=691.2\mathrm{nm}$, 
and the peak of transmission spectrum is $\lambda=1550.55\mathrm{nm}$. 
Figure \ref{fig3.7} illustrates the far-field radiation pattern of upward 
at $\Lambda=691.2\mathrm{nm}$ and $\lambda=1550.55\mathrm{nm}$, 
and the diffracttion angle is $3.5^\circ$, $T=6.22\times10^{-7}$.
\begin{figure}[H]
    \centering
     \includegraphics[width=0.45\textwidth]{img/fig3.6.png}
     \caption{Upward transmission spectrum at $\Lambda=691.2\mathrm{nm}$}
     \label{fig3.6}
\end{figure}
\begin{figure}[H]
    \centering
     \includegraphics[width=0.45\textwidth]{img/fig3.7.png}
     \caption{Far-field radiation pattern of upward at $\Lambda=691.2\mathrm{nm}$,
     $\lambda=1550.55\mathrm{nm}$}
     \label{fig3.7}
\end{figure}
\pagebreak
%=================================================================================================

%=================================================================================================
%Appendix A
%-------------------------------------------------------------------------------------------------
% \section*{\uline{APPENDIX A}}
% %-------------------------------------------------------------------------------------------------
% Here comes some text. This text makes use of 1.5 line spacing. %=================================================================================================
% \pagebreak
%=================================================================================================
% %Appendix B
% %-------------------------------------------------------------------------------------------------
% \section*{\uline{APPENDIX B}}
% %-------------------------------------------------------------------------------------------------
% Here comes some text. This text makes use of 1.5 line spacing.\\
% In order to cite use \cite{Bern:1}, \cite{Bez:1}, \cite{BioModels}, \cite{Row:1} %=================================================================================================
% \pagebreak
% %=================================================================================================
% \bibliographystyle{plain}
% \bibliography{References}
%=================================================================================================

\end{document}